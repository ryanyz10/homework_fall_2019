\documentclass[11pt]{article}
\usepackage{geometry}                
\geometry{letterpaper}
\usepackage[]{graphicx}
\usepackage{amssymb}
\usepackage{amsmath}
\usepackage{braket}
\usepackage{fancyhdr}
\usepackage{qcircuit}
\usepackage{enumitem}
\usepackage{tabu}

\begin{document}

\providecommand{\setnr}{Homework 1}
\providecommand{\due}{Due Thursday, September 12th at 11:59 PM}
\providecommand{\name}{Ryan Zhou (rz3974)}
\newif\ifsols

% swap these to turn solutions on or off
\solstrue
%\solsfalse

%%%%%%%%%%%%%% formatting stuff
\lhead{CS378H, MA375T, PHY341, ES377}
\rhead{\setnr\ \ifsols Solutions \fi}
\lfoot{\due} \cfoot{} \rfoot{Page \thepage}
\renewcommand{\footrulewidth}{0.4pt}
\pagestyle{fancy}

\newcounter{Questions} \newcounter{Parts}

\newcommand{\namequestion}[1]{
\stepcounter{Questions} \setcounter{Parts}{0} 
\vspace{6mm} \noindent {\Large \textbf{#1} \hspace{1mm}}}

\newcommand{\namepart}[1]{
\stepcounter{Parts}
\vspace{2mm} \noindent {\textbf{#1}}}

\newcommand{\question}{\namequestion{\arabic{Questions}.}}
\renewcommand{\part}{\namepart{\alph{Parts}) }}

\newcommand{\sol}[1]{\ifsols \begin{em}  Solution. #1 \end{em} \fi}

\newcommand{\pare}[1]{\left(#1 \right)}
\newcommand{\separrow}{ \hspace{0.5cm}\rightarrow\hspace{0.5cm} }
\newcommand{\sepsemi}{; \hspace{0.5cm}}

%%%%%%%%%%%%%% Document starts here
\begin{center}
    {\Large Homework 1 \ifsols Solutions \fi}
\vspace{2mm}
{\large \\ Introduction to Quantum Information Science}
{\large \\ \ifsols \name \else \due \fi}
\end{center}

\question \textbf{Stochastic and Unitary Matrices.} 

\part The solutions are listed below:
\begin{center}

\begin{tabular} {| c || c | c | c | c | c | c | c | c |} 
    \hline
    & A & B & C & D & E & F & G & H\\
    \hline\hline
    stochastic&&\checkmark&\checkmark&&&&&\\
    \hline
    unitary&&\checkmark&&\checkmark&&\checkmark&\checkmark&\\
    \hline
\end{tabular}

\end{center}

% FIXME this proof is incorrect
\part Suppose $A$ is a $N\times N$ matrix that is both stochastic and unitary. Since $A$ is stochastic, 
all elements of $A$ $a_{ij}\in[0,1]$. Additionally, since all $a_{ij}\in\mathbb{R}$, $A^H=A^T$. Since 
$A$ is unitary, we can deduce that $A^H=A^T=A^{-1},$ or $AA^T=AA^{-1}=I.$ Let $a_i$ be the 
$i$th row vector of $A$. Then $a_ia_i^T=1$ for all $i\in[1,N]$. Since $A$ is stochastic, we also know that
$1a_i^T=1$, where the first $1$ represents a $1\times N$ row vector with all components set to $1$. We can
then subtract these two equations to get that $(1-a_i)a_i^T=0$, which implies that each element $a_{ij}$ of
$a_i$ is either $0$ or $1$. Since $1a_i^T=1$, exactly one component in $a_i$ must be equal to $1$, and the rest
must be $0$. Then $A$ must be a permutation matrix.

\part The matrix 
$\left(\begin{array}
    [c]{cc}%
    0 & 1\\
    1 & 0
\end{array}\right)$
preserves the $4$-norm of real vectors.

\part [Extra credit] From the lead up and also a bit of messing around with $2\times2$ matrices, it seems like only permutation matrices preserve $4$-norms. This seems to make sense since
there also aren't too many permutation matrices compared to unitary or stochastic matrices, and at least one of them ($I$) is trivial. This is far from rigorous though.

\question \textbf{Tensor Products.}

\part $\left(
\begin{array}
[c]{c}%
\frac{2}{3}\\
\frac{1}{3}%
\end{array}
\right)  \otimes\left(
\begin{array}
[c]{c}%
\frac{1}{5}\\
\frac{4}{5}%
\end{array}
\right)=\left(
\begin{array}
[c]{c}%
\frac{2}{15}\\
\frac{8}{15}\\
\frac{1}{15}\\
\frac{4}{15}%
\end{array}
\right)$.


\part Only D cannot be factorized. The rest of the vector factorizations are below:

$A=\left(
\begin{array}
[c]{c}%
\frac{2}{9}\\
\frac{1}{9}\\
\frac{4}{9}\\
\frac{2}{9}%
\end{array}
\right)=\left(
\begin{array}
[c]{c}%
\frac{2}{3}\\
\frac{1}{3}%
\end{array}\right)\otimes\left(
\begin{array}
[c]{c}%
\frac{1}{5}\\
\frac{4}{5}%
\end{array}\right)$

$B=\left(
\begin{array}
[c]{c}%
0\\
1\\
0\\
0%
\end{array}
\right)=\left(
\begin{array}
[c]{c}%
1\\
0%
\end{array}\right)\otimes\left(
\begin{array}
[c]{c}%
0\\
1%
\end{array}\right)$

$C=\left(
\begin{array}
[c]{c}%
\frac{1}{4}\\
\frac{1}{4}\\
\frac{1}{4}\\
\frac{1}{4}%
\end{array}
\right)=\left(
\begin{array}
[c]{c}%
\frac{1}{2}\\
\frac{1}{2}%
\end{array}\right)\otimes\left(
\begin{array}
[c]{c}%
\frac{1}{2}\\
\frac{1}{2}%
\end{array}\right)$

$E=\left(
\begin{array}
[c]{c}%
0\\
\frac{1}{2}\\
0\\
\frac{1}{2}%
\end{array}
\right)=\left(
\begin{array}
[c]{c}%
\frac{1}{\sqrt{2}}\\
\frac{1}{\sqrt{2}}%
\end{array}\right)\otimes\left(
\begin{array}
[c]{c}%
0\\
\frac{1}{\sqrt{2}}%
\end{array}\right)$

\question Suppose there exists a real-valued matrix $A$ such that $A^2=\left(\begin{array}[c]{cc} 1 & 0 \\ 0 & -1\end{array}\right)$. Let
$A=\left(\begin{array}[c]{cc} a_1 & a_2 \\ a_3 & a_4\end{array}\right)$. Then \[A^2=\left(
\begin{array}[c]{cc} a_1^2+a_2a_3 & a_2(a_1+a_4) \\ a_3(a_1+a_4) & a_4^2+a_2a_3 \end{array}\right)=\left(\begin{array}[c]{cc} 1 & 0 \\ 0 & -1\end{array}\right).\]
Then we have the system of equations
\begin{align*}
    &a_1^2+a_2a_3=1\\
    &a_2(a_1+a_4)=0\\
    &a_3(a_1+a_4)=0\\
    &a_4^2+a_2a_3=-1.
\end{align*}
Subtracting the first and last equations gives us $a_1^2-a_4^2=(a_1+a_4)(a_1-a_4)=2$. Clearly, $a_1+a_4\not=0$, thus $a_2=a_3=0$. Then $a_4^2=-1$, which implies that
$a_4\not\in\mathbb{R}$. We conclude that no such matrix $A$ can exist.

\question \textbf{Dirac notation.} 
 
\part $\braket{\psi|\phi}=\frac{2i\braket{0|0}+3\braket{0|1}+4i\braket{1|0}+6\braket{1}{1}}{\sqrt{65}}=\frac{2i+6}{\sqrt{65}}.$ 

\part $A=\sqrt{\braket{\phi|\phi}}=\sqrt{(3i\ket{1}-2i\ket{0})(2i\ket{0}-3i\ket{1})}=\sqrt{13}.$

\part First, we show that the vectors are orthonormal. Two vectors are orthonormal if they are both unit vectors and their inner product is $0$. $\braket{i|i}=\frac{1}{2}(\ket{0}-i\ket{1})(\ket{0}+i\ket{1})=1$,
$\braket{-i|-i}=\frac{1}{2}(\ket{0}+i\ket{1})(\ket{0}-i\ket{1})=1$ and $\braket{i|-i}=\frac{1}{2}(\ket{0}-i\ket{1})(\ket{0}-i\ket{1})=0$. Thus, the vectors are orthonormal. We also have to verify that $\ket{i}$
and $\ket{-i}$ are independent. Suppose there exist scalars $a, b$ such that $a\ket{i}+b\ket{-i}=\frac{1}{\sqrt{2}}((a+b)\ket{0}+i(a-b)\ket{1})=0$. This happens only if $a+b=a-b=0$, or equivalently if $a=b=0$. We
conclude that $\ket{i}$ and $\ket{-i}$ are independent. Then $\ket{i}$ and $\ket{-i}$ form an orthonormal basis for $\mathbb{C}^2$.

\part $\ket{\psi}_{\ket{i},\ket{-i}}=\frac{1}{\sqrt{2}}\left(\begin{array}[c]{cc}1 & 1\\ i & -i\end{array}\right)\ket{\psi}=\frac{1}{\sqrt{26}}\left(\begin{array}[c]{c}-i\\-5\end{array}\right)$

\end{document}

