\documentclass[11pt]{article}
\usepackage{geometry}                
\geometry{letterpaper}
\usepackage[]{graphicx}
\usepackage{amssymb}
\usepackage{amsmath}
\usepackage{braket}
\usepackage{fancyhdr}
\usepackage{qcircuit}

\begin{document}

\providecommand{\setnr}{Homework 4}
\providecommand{\due}{Due Thursday, October 3 at 11:59 PM}
\newif\ifsols

% swap these to turn solutions on or off
%\solstrue
\solsfalse

%%%%%%%%%%%%%% formatting stuff
\lhead{CS378, MA375T, PHY341, ES377}
\rhead{\setnr\ \ifsols Answer Key \fi}
\lfoot{\due} \cfoot{} \rfoot{Page \thepage}
\renewcommand{\footrulewidth}{0.4pt}
\pagestyle{fancy}

\newcounter{Questions} \newcounter{Parts}

\newcommand{\namequestion}[1]{
\stepcounter{Questions} \setcounter{Parts}{0} 
\vspace{6mm} \noindent {\Large \textbf{#1} \hspace{1mm}}}

\newcommand{\namepart}[1]{
\stepcounter{Parts}
\vspace{2mm} \noindent {\textbf{#1}}}

\newcommand{\question}{\namequestion{\arabic{Questions}.}}
\renewcommand{\part}{\namepart{\alph{Parts}) }}

\newcommand{\sol}[1]{\ifsols \begin{em}  Solution. #1 \end{em} \fi}

\newcommand{\pare}[1]{\left(#1 \right)}
\newcommand{\abs}[1]{\left|#1 \right|}
\newcommand{\separrow}{ \hspace{0.5cm}\rightarrow\hspace{0.5cm} }
\newcommand{\sepsemi}{; \hspace{0.5cm}}

%%%%%%%%%%%%%% Document starts here
\begin{center}
    {\Large \setnr\ \ifsols Answer Key \fi}
\vspace{2mm}
{\large \\ Introduction to Quantum Information Science}
{\large \\ \due}
\end{center}

\question We said in class that dense quantum coding requires 1 ebit of
entanglement between Alice and Bob, in addition to 1 qubit of
communication.  In this problem, however, we'll see how to do a ``poor
man's" dense quantum coding with no entanglement, just 1 qubit of
communication from Alice to Bob.

Suppose Alice knows two bits, $x$ and $y$.  She'd like to send a single
qubit to Bob, which will let Bob learn either bit of his choice, $x$ or
$y$, though not necessarily both of them (and she doesn't know which Bob
is interested in).

\part Describe a protocol that lets Bob learn the bit of his choice with
$\cos^2(\pi/8) \approx 85\%$ success probability.
\textit{Hint: You might find the following states useful:}
$$\cos(\pi/8)\ket{0}+\sin(\pi/8)\ket{1}\sepsemi \sin(\pi/8)\ket{0}+\cos(\pi/8)\ket{1}$$$$\cos(\pi/8)\ket{0}-\sin(\pi/8)\ket{1}\sepsemi \sin(\pi/8)\ket{0}-\cos(\pi/8)\ket{1}$$

\part Now suppose Alice is limited to classical communication only.  And
suppose also that the bits x and y are uniformly random and
independent of each other.  Describe a protocol that lets Bob learn
the bit of his choice with 75\% success probability (regardless of which bit he choses to learn).

\clearpage
\question A ``qutrit" has the form $a\ket{0}+b\ket{1}+c\ket{2}$, where $|a|^2+|b|^2+|c|^2=1$.
Suppose Alice and Bob share the entangled state $(\ket{00}+\ket{11}+\ket{22})/\sqrt{3}$.  Then consider the following protocol for teleporting a qutrit $\ket{\psi}=a\ket{0}+b\ket{1}+c\ket{2}$ from Alice to Bob: first
Alice applies a CSUM gate from $\ket{\psi}$ onto her half of the entangled pair, where
$$\text{CSUM} (\ket{x} \otimes \ket{y}) = \ket{x} \otimes \ket{y + x\text{ }(\text{mod } 3)}.$$
\noindent Next, Alice applies the unitary matrix $F$ to the $\ket{\psi}$ register, where
$$ F =\frac{1}{\sqrt{3}} \begin{bmatrix} 1&1&1 \\ 1& \omega & \omega^2 \\ 1 & \omega^2 & \omega \end{bmatrix},$$
\noindent and $\omega = e^{2i\pi/3}$ so that $\omega^3 = 1$. She then measures both of her qutrits in the
$\{\ket{0},\ket{1},\ket{2}\}$ basis, and sends the results to Bob over a classical channel. Show that Bob can recover a local copy of $\ket{\psi}$ given these measurement results. 

This quantum circuit summarizes the protocol:

\[
\Qcircuit @C=.5em @R=0.5em @!R {
	& \lstick{\ket{\psi}} & \multigate{1}{\text{CSUM}} & \gate{F} & \meter \cwx[2] \\
	& \multigate{1}{\frac{\ket{00}+\ket{11}+\ket{22}}{\sqrt{3}}} &  \ghost{\text{CSUM}} & \meter \cwx[1] \\
	& \ghost{\frac{\ket{00}+\ket{11}+\ket{22}}{\sqrt{3}}} & \qw & \gate{?}  & \gate{?} & \qw & \rstick{\ket{\psi_\text{out}}} \\
}
\]

Here the double-lines represent classical `trits' being sent from Alice to Bob. Depending on the value of the first trit he receives, 0,1 or 2, Bob can apply one of three corrective gates(Where ``?'' is a placeholder for the three gates which you will have to determine). Depending on the value of the second trit he receives he can apply a particular corrective gate either 0, 1 or 2 times (Where ``?'' is a placeholder for this gate, possibly different than any used for the first trit, which you will determine). Prove that $\ket{\psi} = \ket{\psi_\text{out}}$ for appropriately chosen $?$ gates for all possible measurement results. \textit{Hint: You \emph{could} explicitly work out all 9 possible cases, but you
	could also save time by noticing a general pattern that lets you
	handle all the cases in a unified way.}

\question Suppose Alice and Bob hold one qubit each of an arbitrary two-qubit state $\ket{\psi}$ that is possibly entangled. They can apply local operations and are allowed classical communication. Their goal is to apply the CNOT gate to their state $\ket{\psi}$. Describe a protocol they can use to achieve this using two ebits of entanglement. 

\question In the ``GHZ game," there are three players, Alice, Bob, and
Charlie, who are given bits $x$, $y$, and $z$ respectively.  We're promised
that $x+y+z=0 \hspace{.5em} \text{(mod 2)}$; otherwise the bits could be arbitrary.  The
players' goal is, without communicating with each other, to output
bits $a$, $b$, $c$ respectively such that
$a+b+c \text{(mod 2)} = \text{OR}(x,y,z)$.
In other words, the parity of their output bits should be odd, if and
only if at least one of the input bits is nonzero.

\part Show that, in a classical universe, there is no strategy that
causes the players to succeed with certainty, for all four possible
allowed inputs $(x,y,z)$.

\part Now suppose that the players share the state:

$$\frac{\ket{000} - \ket{011} - \ket{101} - \ket{110}}{2}$$  

Suppose that each player measures their
qubit in the $\{\ket{0}, \ket{1}\}$ basis if their input bit is 0, or in the
$\{\ket{+}, \ket{-}\}$ basis if their input bit is 1, and that they output
according to what they see.  Show that this lets the players win the
GHZ game with certainty, for all four possible input triples.

\part \textbf{Extra credit}. Design a protocol that works for the GHZ state:

$$\frac{\ket{000} + \ket{111}}{\sqrt{2}}$$  


\end{document}

