\documentclass[11pt]{article}
\usepackage{geometry}                
\geometry{letterpaper}
\usepackage[]{graphicx}
\usepackage{amssymb}
\usepackage{amsmath}
\usepackage{braket}
\usepackage{fancyhdr}
\usepackage{qcircuit}

\begin{document}

\providecommand{\setnr}{Homework 2}
\providecommand{\due}{Due Thursday, September 19\textsuperscript{th} at 11:59 PM}
\providecommand{\name}{Ryan Zhou (rz3974)}
\newif\ifsols

% swap these to turn solutions on or off
\solstrue
%\solsfalse

%%%%%%%%%%%%%% formatting stuff
\lhead{CS378H, MA375T, PHY341, ES377}
\rhead{\setnr\ \ifsols Answer Key \fi}
\lfoot{\due} \cfoot{} \rfoot{Page \thepage}
\renewcommand{\footrulewidth}{0.4pt}
\pagestyle{fancy}

\newcounter{Questions} \newcounter{Parts}

\newcommand{\namequestion}[1]{
\stepcounter{Questions} \setcounter{Parts}{0} 
\vspace{6mm} \noindent {\Large \textbf{#1} \hspace{1mm}}}

\newcommand{\namepart}[1]{
\stepcounter{Parts}
\vspace{2mm} \noindent {\textbf{#1}}}

\newcommand{\question}{\namequestion{\arabic{Questions}.}}
\renewcommand{\part}{\namepart{\alph{Parts}) }}

\newcommand{\sol}[1]{\ifsols \begin{em}  Solution. #1 \end{em} \fi}

\newcommand{\pare}[1]{\left(#1 \right)}
\newcommand{\separrow}{ \hspace{0.5cm}\rightarrow\hspace{0.5cm} }
\newcommand{\sepsemi}{; \hspace{0.5cm}}

%%%%%%%%%%%%%% Document starts here
\begin{center}
    {\Large \setnr\ \ifsols Solutions \fi}
\vspace{2mm}
{\large \\ Introduction to Quantum Information Science}
{\large \\ \ifsols \name \else \due \fi}
\end{center}

\question \textbf{More fun with matrices.}

\part The $2\times2$ unitary matrix is straightforward:
\[\left(\begin{array}
    [c]{cc}%
    0 & -1\\
    1 & 0
\end{array}\right).\]

\part A $4\times4$ unitary matrix satisfying the constraints is below:
\[\left(\begin{array}
    [c]{cccc}%
    0 & \frac{1}{\sqrt{3}} & \frac{1}{\sqrt{3}} & \frac{1}{\sqrt{3}}\\
    \frac{1}{\sqrt{3}} & 0 & \frac{1}{\sqrt{3}} & -\frac{1}{\sqrt{3}}\\
    -\frac{1}{\sqrt{3}} & \frac{1}{\sqrt{3}} & 0 & -\frac{1}{\sqrt{3}}\\
    \frac{1}{\sqrt{3}} & \frac{1}{\sqrt{3}} & -\frac{1}{\sqrt{3}} & 0
\end{array}\right).\]

\part Let $A=
\left(\begin{array}
    [c]{ccc}%
    0 & c & e\\
    a & 0 & f\\
    b & d & 0
\end{array}\right)
$ be a $3\times3$ unitary matrix with only diagonal $0$ Since $A$ is unitary, $A^HA=I$. We then have the equations
\begin{align*}
    |a|^2 + |b|^2 = 1 &\qquad \overline{a}f=a\overline{f}=0 \\
    |c|^2 + |d|^2 = 1 &\qquad \overline{b}d=b\overline{d}=0 \\
    |e|^2 + |f|^2 = 1 &\qquad \overline{c}e=c\overline{e}=0.
\end{align*}
Then one of each pair $(a,f), (b,d), (c,e)$ must be $0$. This implies that there exist non-diagonal entries in $A$ that are $0$, a contradiction.
Thus, no such matrix $A$ can exist.

\question \textbf{Single Qubit Quantum Circuits.} 

\part $SHZH\ket{0}=SHZ\ket{+}=SH\ket{-}=S\ket{1}=i\ket{1}=\ket{1}$. If we measure in the ${\ket{0},\ket{1}}$ basis, $P[\ket{0}]=0$, $P[\ket{1}]=1$.

\part $HYZR_{\pi/4}\ket{0}=HYZ\ket{+}=HY\ket{-}=H(iXZ\ket{-})=H(iX\ket{+})=H(i\ket{+})=H\ket{+}=\ket{0}.$ If we measure in the ${\ket{+},\ket{-}}$ basis,
$P[\ket{+}]=\frac{1}{2}$, $P[\ket{-}]=\frac{1}{2}$.

\part $HT\ket{+}=H(\frac{1}{\sqrt{2}}\ket{0}+\frac{1}{\sqrt{2}}e^{i\pi/4}\ket{1})=\frac{1}{\sqrt{2}}\ket{+}+\frac{1}{\sqrt{2}}e^{i\pi/4}\ket{-}$.

\part $TZT\ket{+}=TZ(\frac{1}{\sqrt{2}}\ket{0}+\frac{1}{\sqrt{2}}e^{i\pi/4}\ket{1})=T(\frac{1}{\sqrt{2}}\ket{0}-\frac{1}{\sqrt{2}}e^{i\pi/4}\ket{1})=\frac{1}{\sqrt{2}}\ket{0}-\frac{1}{\sqrt{2}}e^{i\pi/2}\ket{1}.$

\clearpage

\question \textbf{Miscellaneous.}

\part Intuitively, $\ket{0}+\ket{+}$ is the superposition of the $\ket{0}$ and $\ket{+}$ states. Let $\ket{\phi}=\ket{0}+\ket{1}$. Note that $\braket{\phi|\phi}=2+\sqrt{2}$. Then $\ket{\psi}=\frac{1}{\sqrt{2+\sqrt{2}}}\ket{\phi}$.

\part $H\ket{\psi}=\frac{1}{\sqrt{2+\sqrt{2}}}H(\ket{0}+\ket{+})=\frac{1}{\sqrt{2+\sqrt{2}}}(\ket{+}+\ket{0})=\ket{\psi}$. Then $\ket{\psi}$ is an eigenstate of the $H$ gate with $\lambda=1$.

\part There are $8$ different states reachable from $\ket{0}$ with $H$ and $S$:
\begin{align*}
    \ket{0} & \qquad H\ket{0}=\ket{+}\\
    SH\ket{0}=\ket{i} & \qquad HSH\ket{0}=H\ket{i}=\frac{\ket{+}+i\ket{-}}{\sqrt{2}}\\
    S^2H\ket{0}=\ket{-} & \qquad HS^2H\ket{0}=H\ket{-}=\ket{1}\\
    s^3H\ket{0}=\ket{-i} & \qquad HS^3H\ket{0}=H\ket{-i}=\frac{\ket{+}-i\ket{-}}{\sqrt{2}}.
\end{align*}

\question \textbf{Distinguishability of states.} 

\part Consider the quantum circuit: 
\[
\Qcircuit @C=1em @R=.7em {
    \lstick{\ket{\psi}} & \gate{R_{\pi/8}} & \meter
} 
\]
If $\ket{\psi}=\ket{0}$, $R_{\pi/8}\ket{\psi}=\cos{(\frac{\pi}{8})}\ket{0}+\sin{(\frac{\pi}{8})}\ket{1}$, so probability of failure is $\sin^2{(\pi/8)}$. If $\ket{\psi}=\ket{+}$, $R_{\pi/8}\ket{\psi}=\cos{(\frac{3\pi}{8})}\ket{0}+\sin{(\frac{3\pi}{8})}\ket{1}$, so probability of failure is $\cos^2{(3\pi/8)}=\sin^2{(\pi/8)}$. Then overall probability of failure is $\sin^2(\frac{\pi}{8})$.\\
\textbf{Extra Credit:} Suppose that we rotate by some angle $\epsilon\in[0,\frac{\pi}{4}]$. If $\ket{\psi}=\ket{0}$, $R_\epsilon\ket{\psi}=\cos{\epsilon}\ket{0}+\sin{\epsilon}\ket{1}$, and if $\ket{\psi}=\ket{+},R_\epsilon\ket{\psi}=\cos{(\frac{\pi}{4}+\epsilon)}\ket{0}+\sin{(\frac{\pi}{4}+\epsilon)}\ket{1}$. Then the overall probability of failure $p_f=\sin^2{(\epsilon)}+\cos^2{(\frac{\pi}{4}+\epsilon)}$. The first derivative of this is
$p_f'=2\sin(\epsilon)\cos(\epsilon)-2\sin(\frac{\pi}{4}+\epsilon)\cos(\frac{\pi}{4}+\epsilon)=0$. Rearranging and converting the $\sin$ to $\cos$, $\cos(\frac{\pi}{2}-\epsilon)\cos(\epsilon)=\cos(\frac{\pi}{4}-\epsilon)\cos(\frac{\pi}{4}+\epsilon).$ Then either 
\begin{align*}
    \frac{\pi}{2}-\epsilon=\frac{\pi}{4}-\epsilon,\quad\epsilon=\frac{\pi}{4}+\epsilon,
\end{align*}
or
\begin{align*}
    \frac{\pi}{2}-\epsilon=\frac{\pi}{4}+\epsilon,\quad\epsilon=\frac{\pi}{4}-\epsilon.
\end{align*}
The first case is impossible, but in the second case we find that optimally $\epsilon=\frac{\pi}{8}$.

\part The probability of failure is equal to the probability of getting $\ket{0}$ and measuring $1$ plus the probability of getting $\ket{+}$ and measuring $0$. 
Then $P[\text{failure}]=\frac{1}{2}0+\frac{1}{2}\frac{1}{2}=\frac{1}{4}$.


\end{document}

