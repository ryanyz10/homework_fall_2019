\documentclass[11pt]{article}
\usepackage{geometry}                
\geometry{letterpaper}
\usepackage[]{graphicx}
\usepackage{amssymb}
\usepackage{amsmath}
\usepackage{braket}
\usepackage{fancyhdr}
\usepackage{qcircuit}

\begin{document}

\providecommand{\setnr}{Problem Set \#3}
\providecommand{\due}{Due Thursday, September 26\textsuperscript{th} at 11:59 PM}
\providecommand{\name}{Ryan Zhou (rz3974)}
\newif\ifsols

% swap these to turn solutions on or off
\solstrue
%\solsfalse

%%%%%%%%%%%%%% formatting stuff
\lhead{CS378H, MA375T, PHY341}
\rhead{\setnr\ \ifsols Solutions \fi}
\lfoot{\due} \cfoot{} \rfoot{Page \thepage}
\renewcommand{\footrulewidth}{0.4pt}
\pagestyle{fancy}

\newcounter{Questions} \newcounter{Parts}

\newcommand{\namequestion}[1]{
\stepcounter{Questions} \setcounter{Parts}{0} 
\vspace{6mm} \noindent {\Large \textbf{#1} \hspace{1mm}}}

\newcommand{\namepart}[1]{
\stepcounter{Parts}
\vspace{2mm} \noindent {\textbf{#1}}}

\newcommand{\question}{\namequestion{\arabic{Questions}.}}
\renewcommand{\part}{\namepart{\alph{Parts}) }}

\newcommand{\sol}[1]{\ifsols \begin{em}  Solution. #1 \end{em} \fi}

\newcommand{\pare}[1]{\left(#1 \right)}
\newcommand{\separrow}{ \hspace{0.5cm}\rightarrow\hspace{0.5cm} }
\newcommand{\sepsemi}{; \hspace{0.5cm}}

%%%%%%%%%%%%%% Document starts here
\begin{center}
    {\Large \setnr\ \ifsols Solutions \fi}
\vspace{2mm}
{\large \\ Introduction to Quantum Information Science}
{\large \\ \ifsols \name \else \due \fi}
\end{center}

\question \textbf{Local Evolution of Entangled States.} 

Let $U=\left(\begin{array}[c]{cc}a & b\\c & d\\\end{array}\right)$ and $U^T=\left(\begin{array}[c]{cc}a & c\\b & d\\\end{array}\right).$ Then
\begin{align*}
    (U\otimes I)\ket{\psi}&=\frac{1}{\sqrt{2}}\left[\left(\begin{array}[c]{c}a\\c\end{array}\right)\otimes\left(\begin{array}[c]{c}1\\0\end{array}\right)
    +\left(\begin{array}[c]{c}b\\d\end{array}\right)\otimes\left(\begin{array}[c]{c}0\\1\end{array}\right)\right]\\
                          &=\frac{1}{\sqrt{2}}\left[\left(\begin{array}[c]{c}a\\0\\c\\0\end{array}\right)+\left(\begin{array}[c]{c}0\\b\\0\\d\end{array}\right)\right]\\
                          &=\frac{1}{\sqrt{2}}\left(\begin{array}[c]{c}a\\b\\c\\d\end{array}\right)=
                          \frac{1}{\sqrt{2}}\left[\left(\begin{array}[c]{c}a\\b\\0\\0\end{array}\right)+\left(\begin{array}[c]{c}0\\0\\c\\d\end{array}\right)\right]\\
                          &=\frac{1}{\sqrt{2}}\left[\left(\begin{array}[c]{c}1\\0\end{array}\right)\otimes\left(\begin{array}[c]{c}a\\b\end{array}\right)
    +\left(\begin{array}[c]{c}0\\1\end{array}\right)\otimes\left(\begin{array}[c]{c}c\\d\end{array}\right)\right]\\
    &=(I\otimes U^T)\ket{\psi}.
\end{align*}

\question \textbf{Multi-qubit quantum circuits}

\part Writing the circuit in terms of matrix multiplication, we have
\begin{align*}
(H\otimes H)\text{cX}(H\otimes H)&=\frac{1}{4}\left(\begin{array}[c]{cccc}1&1&1&1\\1&-1&1&-1\\1&1&-1&-1\\1&-1&-1&1\end{array}\right)
    \left(\begin{array}[c]{cccc}1&0&0&0\\0&1&0&0\\0&0&0&1\\0&0&1&0\end{array}\right)
    \left(\begin{array}[c]{cccc}1&1&1&1\\1&-1&1&-1\\1&1&-1&-1\\1&-1&-1&1\end{array}\right)\\
    &=\frac{1}{4}\left(\begin{array}[c]{cccc}1&1&1&1\\1&-1&1&-1\\1&1&-1&-1\\1&-1&-1&1\end{array}\right)
    \left(\begin{array}[c]{cccc}1&1&1&1\\1&-1&1&-1\\1&-1&-1&1\\1&1&-1&-1\end{array}\right)\\
    &=\frac{1}{4}\left(\begin{array}[c]{cccc}4&0&0&0\\0&0&0&4\\0&0&4&0\\0&4&0&0\end{array}\right)\\
    &=\left(\begin{array}[c]{cccc}1&0&0&0\\0&0&0&1\\0&0&1&0\\0&1&0&0\end{array}\right).
\end{align*}
Note that this final array is a reverse cX gate with the $2$nd q-bit is used as the control.

\vspace{1em}
\part Let $\ket{\psi}=\alpha\ket{0}+\beta\ket{1}$. The answers to the questions are below:
\begin{enumerate}
    \item What is the state of the first qubit before the CNOT? $\frac{1}{\sqrt{2}}\left(\begin{array}[c]{c}1\\e^{i\pi/4}\end{array}\right).$
    \item What is the state of the two qubits before the measurement? $\frac{1}{\sqrt{2}}\left(\begin{array}[c]{c}\beta\\\alpha e^{i\pi/4}i\\\beta e^{i\pi/4}\\\alpha i\end{array}\right).$
    \item What are the probabilities of measuring $\ket{0}$ or $\ket{1}$? $P[\ket{0}]=\frac{1}{2}(\beta^2+\alpha^2)=\frac{1}{2},P[\ket{1}]=\frac{1}{2}(\alpha^2+\beta^2)=\frac{1}{2}$.
    \item What is the second qubit state $\ket{\psi_\text{out}}$ when the first qubit is measured as $\ket{0}$. How about when it's measured as $\ket{1}$? If the first qubit is $\ket{0}$, $\ket{\psi_\text{out}}=\beta\ket{0}+\alpha e^{i\pi/4}i\ket{1}$. If the first qubit is $\ket{1}$, $\ket{\psi_\text{out}}=\beta e^{i\pi/4}\ket{0}+\alpha i\ket{1}$.
\end{enumerate}

\part The circuit is below:
\[
\Qcircuit @C=.5em @R=0.0em @!R {
&  & \ctrl{2} &  \qw&  &  &  & \ctrl{2} & \qw\\
& & & & & \push{\rule{.3em}{0em}=\rule{.3em}{0em}}\\   
&  \gate{H}& \targ & \gate{H} & \qw &  & & \gate{Z} & \qw \\}
\]

\question \textbf{Quantum Money Attacks.} If we expand all the tensor products, we get the giant vector
\[
    \left(\begin{array}[c]{c}
        \frac{\sqrt{3}}{2}\\0\\0\\\frac{1}{\sqrt{12}}\\0\\\frac{1}{\sqrt{12}}\\\frac{1}{\sqrt{12}}\\0
    \end{array}\right).
\]

Then the probability of success for any given qubit state is $P(\text{success})=(\frac{5}{6})(\frac{9}{10})+(\frac{1}{6})(0)=\frac{3}{4}$. Note that each case
is symmetric, thus the overall probability is $4(\frac{1}{4})(\frac{3}{4})=\frac{3}{4}>\frac{5}{8}$.

\question \textbf{SARG04 Quantum Key Distribution}

\part $a=011001$, $b=101011$, $\ket{\psi}=\ket{+}\ket{1}\ket{-}\ket{0}\ket{+}\ket{-}$.

\part $b'=100111$, $\ket{\psi'}=\ket{+}\ket{1}\ket{0}\ket{-}\ket{+}\ket{-}$, $a'=010101$.

\part Alice and Bob will discard indices $2$ and $3$, so $a=a'=0101$.

\part Assign $\{\ket{0}, \ket{+}\}=00$, $\{\ket{0}, \ket{-}\}=01$, $\{\ket{1}, \ket{+}\}=10$ or $\{\ket{1}, \ket{-}\}=11$. Then we can encode $\ket{psi}$
with the classical string $101111001001$.

\part Using $a'$ from part b, we get a $b'=1?1111$.

\part Alice and Bob will discard indices $1$ and $2$, so $a=0001$ and $a'=0101$. I'm not sure if $a,a'$ are supposed to match in this scheme. It seems like they should, but I didn't really understand the instructions we were given for SARG04.

\end{document}

